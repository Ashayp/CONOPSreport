\chapter{Justification for and nature of changes \\ 
% \small{\textit{-- Author Name}}
\index{Justification for and nature of changes}
\label{Chapter::Justificationforandnatureofchanges}} 
\section{Justication of changes \label{Section::Justicationofchanges}}
The justification for these changes is to address several key issues identified through a recent analysis of our company's operations. This analysis revealed a number of inefficiencies and bottlenecks that have been negatively impacting productivity and profitability. The proposed changes aim to streamline workflows, improve communication and collaboration, and optimize resource allocation to achieve better results.

\noindent
There are several reasons why changes are required, including:

\begin{enumerate}

    \item New business requirements: The business requirements for the instore inventory management system may change as the business grows or as market conditions change. These changes may require modifications to the system to support new requirements.

    \item Technological advancements: Advancements in technology may provide new opportunities to improve the system's performance, reliability, and scalability. These advancements may require changes to the system to take advantage of the new capabilities.

    \item User feedback: Feedback from users may reveal areas where the system could be improved to better meet their needs. This feedback may prompt changes to the system's design or functionality.

    \item Regulatory requirements: Changes in regulatory requirements may require modifications to the system to ensure compliance with new regulations.

    \item External factors: External factors, such as changes in the competitive landscape or economic conditions, may require modifications to the system to remain competitive or to reduce costs.

\end{enumerate}

The nature of changes in the instore inventory management system can range from minor modifications to major overhauls. Changes may involve adding new functionality, modifying existing functionality, or removing functionality that is no longer required. Changes may also involve modifications to the system's architecture or infrastructure to improve performance, reliability, or scalability.


\section{Description of desired changes \label{Section::Descriptionofdesiredchanges}}
The instore inventory management system requires several changes to enhance its performance, reliability, and scalability. Some of the desired changes are:

\begin{enumerate}

    \item Multi-store support: The system will now support multiple stores, enabling store managers and staff to manage inventory across several locations. This change will provide centralized inventory control and enable inventory transfers between stores, reducing the risk of stockouts or overstocking.

    \item Mobile access: The system will now provide mobile access, allowing store managers and staff to access inventory data and perform inventory management tasks from anywhere at any time. This change will enable remote inventory management and reduce the need for physical presence in the store, increasing efficiency and reducing costs.

    \item Customizable alerts: The system will now enable customizable alerts that notify store managers and staff of low inventory levels, stockouts, and other inventory-related issues. These alerts can be customized to specific thresholds and sent via email, SMS, or in-app notifications. This change will help ensure timely inventory management and reduce the risk of stockouts or overstocking.

    \item Integration with POS system: The system will now integrate with the point-of-sale (POS) system, allowing inventory levels to be automatically updated based on sales transactions. This change will ensure accurate inventory tracking and reduce errors related to manual data entry.

    \item Vendor management: The system will now include vendor management functionality, enabling store managers and staff to manage vendor relationships and track vendor performance. This change will help optimize the purchasing process and ensure timely delivery of goods, reducing the risk of stockouts or overstocking.
    
\end{enumerate}
These changes will improve the system's functionality and make it more user-friendly, efficient, and accurate. The system will provide multi-store support, mobile access, customizable alerts, integration with the POS system, and vendor management functionality, enabling store managers and staff to manage inventory efficiently and effectively across multiple stores.

\section{Priorities among changes \label{Section::Prioritiesamongchanges}}
The priorities of the desired changes for the inventory management system would depend on the specific needs of the product. However, general benefits are:

\begin{enumerate}

    \item Integration with POS system: Integrating the inventory management system with the point-of-sale (POS) system should be a top priority as it would enable automatic inventory updates based on sales transactions. This would ensure accurate inventory tracking, reduce errors related to manual data entry, and provide real-time visibility into inventory levels.

    \item Customizable alerts: Implementing customizable alerts should be a high priority as it would notify store managers and staff of low inventory levels, stockouts, and other inventory-related issues, enabling timely inventory management and reducing the risk of stockouts or overstocking.

    \item Mobile access: Providing mobile access to the inventory management system should also be a high priority as it would enable store managers and staff to access inventory data and perform inventory management tasks from anywhere at any time, increasing efficiency and reducing costs.

    \item Vendor management: Including vendor management functionality should be a moderate priority as it would help optimize the purchasing process and ensure timely delivery of goods, reducing the risk of stockouts or overstocking.

    \item Multi-store support: Implementing multi-store support should be a moderate priority as it would enable store managers and staff to manage inventory across multiple locations, providing centralized inventory control and enabling inventory transfers between stores.

\end{enumerate}
Prioritizing these changes would enable the organization to implement the most critical improvements first and progressively enhance the system's functionality and efficiency over time.

\section{Changes considered but not included \label{Section::Changesconsideredbutnotincluded}}
During the development of the proposed instore inventory management system, several changes may have been considered but not included for various reasons. May be in a future iteration these can be incorporated. Some of these changes may include:

\begin{enumerate}
    \item RFID Technology: The use of RFID technology to track inventory in real-time is an option that may have been considered. However, this technology can be expensive to implement, and it may not be feasible for smaller organizations with limited resources.

    \item Advanced Artificial Intelligence: The use of artificial intelligence to automate inventory management tasks and make more accurate predictions have been considered. However, this technology can be complex and may require significant resources to implement and maintain. This will require a certain scale of operation before it may become profitable. Meanwhile, systems with lesser accuracy may suffice.

    \item Automatic Reorder: Automatic reorder functionality can be used to automatically generate purchase orders when inventory levels fall below a certain threshold. However, this feature involves a high risk factor and utilizing this without thorough testing and establishment may taint the products image in the market.

    \item Custom Barcode Generation: Barcode generation functionality can be used to generate barcodes for inventory items. However, This system may not be necessary as most products already come with a unique barcode. It is done only in case of threat with regards to barcode tampering, which only happens when the suppliers are non trusted entities, which is extremely rare.

\end{enumerate}

\newpage
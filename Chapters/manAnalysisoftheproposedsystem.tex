\chapter{Analysis of the proposed system \\ 
% \small{\textit{-- Author Name}}
\index{Analysis of the proposed system}
\label{Chapter::Analysisoftheproposedsystem}}
\section{Summary of improvements \label{Section::Summaryofimprovements}}
\begin{enumerate}
    \item Multi-store support: The proposed system will allow for centralized inventory control across multiple locations, enabling inventory transfers between stores.

    \item Mobile access: The proposed system will provide mobile access, allowing for remote inventory management and reducing the need for physical presence in the store.

    \item Customizable alerts: The proposed system will enable customizable alerts to notify store managers and staff of low inventory levels, stockouts, and other inventory-related issues.

    \item Integration with POS system: The proposed system will integrate with the point-of-sale (POS) system, automatically updating inventory levels based on sales transactions, reducing errors related to manual data entry.

    \item Vendor management: The proposed system will include vendor management functionality, allowing for the optimization of the purchasing process and timely delivery of goods.
\end{enumerate}
\noindent
Overall, the proposed system will improve inventory accuracy, increase efficiency, and reduce costs, making it a worthwhile investment.

\section{Disadvantages and limitations \label{Section::Disadvantagesandlimitations}}
\begin{enumerate}


    \item Cost: Developing and implementing the proposed system will require a significant investment of time and resources, including software development, hardware procurement, and staff training.

    \item Technical challenges: The development and implementation of the proposed system may encounter technical challenges, including software bugs, hardware failures, or integration issues with other systems.

    \item Organizational change: The proposed system may require changes to existing workflows or data management practices, which may be disruptive and require staff training.

    \item Security concerns: Storing inventory data in a centralized system may pose security risks if not properly secured.
\end{enumerate}

\section{Alternatives and trade-offs considered \label{Section::Alternativesandtradeoffsconsidered}}
\begin{enumerate}
    \item Even though our system tries to fill in the gaps that are present in the market, the cost of doing that still remains unless and until the system scales and more retailers join, the cost of the system may remain high.
    \item To initially keep the system simple, we decided to keep english as the primary language used in the system. While, we have seen that many of the users may not be well versed with a particular language. However, this ensures a faster development of the initial product in english while later extending support to other languages.
    \item Targeted use of Cloud increases the maintenance and upfront cost of the system, however, in the long run it ensures that the system has higher availability and better scalability.
    \item Incorporation of the system may require changes to the existing workflow as in the initial software versions are designed with specific workflows in mind. Again to ensure a smoother initial release. This workflow is the one that we found to be the most commonly used.
    \item Incorporation of an advanced prediction system for forecasting inventory was considered but the resources required for it may make the system unfeasible. Thus, it should not be a part of the main system. 
\end{enumerate}
\newpage